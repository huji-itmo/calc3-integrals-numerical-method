\subsection{Численный метод.}
\subsubsection{Результаты вычислений:}

\begin{table}[h!t]
    \centering
    \begin{tabular}{|c|c|c|c|}
    \hline
    $\delta$ & интегральная сумма & отклонение & время выполнения (c)\\
    \hline
    0.1 & -0.035333 & -0.035333 & 0.000272\\
    \hline
    0.01 & -0.00037 & -0.00037 & 0.002479\\
    \hline
    0.001 & -4e-06 & -4e-06 & 0.02521\\
    \hline
\end{tabular}

    \caption{Криволинейный интеграл.}
\end{table}

\begin{table}[h!t]
    \centering
    \begin{tabular}{|c|c|c|c|}
    \hline
    $\delta$ & интегральная сумма & отклонение & время выполнения\\
    \hline
    0.1 & -2.6299 & -2.6299 & 0.010953\\
    \hline
    0.01 & -0.295948 & -0.295948 & 1.194271\\
    \hline
    0.001 & -0.030126 & -0.030126 & 139.606398\\
    \hline
\end{tabular}

    \caption{Двойного интеграл (минимальная сумма).}
\end{table}

\begin{table}[h!t]
    \centering
    \begin{tabular}{|c|c|c|c|}
    \hline
    $\delta$ & интегральная сумма & отклонение & время выполнения\\
    \hline
    0.1 & 3.5532 & 3.5532 & 0.030598\\
    \hline
    0.01 & 0.304213 & 0.304213 & 1.960662\\
    \hline
    0.001 & 0.030197 & 0.030197 & 195.901473\\
    \hline
\end{tabular}

    \caption{Двойного интеграл (максимальная сумма).}
\end{table}

Как мы видим, двойной интеграл очень долго считать для малых значений $\delta$. Возможно, потому что асимптотическая сложность $O((1/\delta)^2) \sim O(n^2) $, в отличие от криволинейного, со сложностью $O(1/\delta) \sim O(n)$

\begin{quote}
  Не так я всё это представлял, когда мы начинали. Мне грезились деньги, машины, женщины, уважение, свобода. Я всё это даже получил, более или менее, но вместе с тем пришли тюрьма, постоянный страх и кровь моих товарищей. Я держался на плаву сколько мог, но шансов становилось всё меньше.  
\end{quote}
\newpage
\subsection{Программный код.}

\underline{\href{https://github.com/huji-itmo/calc3-integrals-numerical-method}{Посмотреть исходный код на Github}}

\subsubsection{main.py}
\verbatiminput{Task10/python/main.py}

\subsubsection{intergral\_configuration.py}
\verbatiminput{Task10/python/integral_config.py}

\subsubsection{double\_integral\_sum.py}
\verbatiminput{Task10/python/double_integral_sum.py}

\subsubsection{vector\_line\_integral\_sum.py}
\verbatiminput{Task10/python/vector_line_integral_sum.py}

\subsubsection{latex.py}
\verbatiminput{Task10/python/latex.py}

\subsubsection{plot.py}
\verbatiminput{Task10/python/plot.py}